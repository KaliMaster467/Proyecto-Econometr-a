\documentclass{article}
\usepackage{amsmath}
\usepackage{amssymb}
\usepackage[utf8]{inputenc}

\begin{document}

\section*{Cálculo del Excedente del Consumidor, Excedente del Productor y Pérdida en Bienestar Social}

Dado el gráfico proporcionado, los cálculos se basan en la curva de demanda inversa, el precio marginal ($P_{CM}$), y las cantidades demandadas correspondientes.

\subsection*{1. Excedente del Consumidor (EC)}
El excedente del consumidor es el área del triángulo comprendido entre el precio inicial ($P_{CM} = 47.8$) y el precio en equilibrio ($P = 68.11$), hasta la cantidad de equilibrio ($Q = 770$). 

\[
EC = \frac{1}{2} \cdot \text{Base} \cdot \text{Altura}
\]

Donde:
\begin{itemize}
    \item \text{Base} = $770 - 0 = 770$ (miles de barriles diarios),
    \item \text{Altura} = $68.11 - 47.8 = 20.31$ (dólares).
\end{itemize}

Sustituyendo:
\[
EC = \frac{1}{2} \cdot 770 \cdot 20.31 = 7,821.35 \, \text{miles de dólares}.
\]

\subsection*{2. Excedente del Productor (EP)}
El excedente del productor es el área del triángulo entre el precio marginal ($P_{CM} = 47.8$) y el precio en equilibrio ($P = 68.11$), hasta la cantidad de equilibrio ($Q = 770$). 

\[
EP = \frac{1}{2} \cdot \text{Base} \cdot \text{Altura}
\]

Donde:
\begin{itemize}
    \item \text{Base} = $770 - 0 = 770$ (miles de barriles diarios),
    \item \text{Altura} = $47.8 - 17.8 = 30$ (dólares).
\end{itemize}

Sustituyendo:
\[
EP = \frac{1}{2} \cdot 770 \cdot 30 = 11,550 \, \text{miles de dólares}.
\]

\subsection*{3. Pérdida en Bienestar Social (PBS)}
La pérdida en bienestar social se encuentra en el área del triángulo que resulta de la cantidad que se pierde al reducir la producción desde $Q_{max} = 1423.25$ hasta $Q_{equilibrio} = 770$.

\[
PBS = \frac{1}{2} \cdot \text{Base} \cdot \text{Altura}
\]

Donde:
\begin{itemize}
    \item \text{Base} = $1423.25 - 770 = 653.25$ (miles de barriles diarios),
    \item \text{Altura} = $68.11 - 47.8 = 20.31$ (dólares).
\end{itemize}

Sustituyendo:
\[
PBS = \frac{1}{2} \cdot 653.25 \cdot 20.31 = 6,629.32 \, \text{miles de dólares}.
\]

\subsection*{Resultados Finales}
\begin{itemize}
    \item Excedente del Consumidor (EC): $7,821.35$ miles de dólares.
    \item Excedente del Productor (EP): $11,550$ miles de dólares.
    \item Pérdida en Bienestar Social (PBS): $6,629.32$ miles de dólares.
\end{itemize}

\end{document}
